\documentclass{beamer}

\mode<presentation>
{
  \usetheme{default}      % or try Darmstadt, Madrid, Warsaw, ...
  \usecolortheme{default} % or try albatross, beaver, crane, ...
  \usefonttheme{default}  % or try serif, structurebold, ...
  \setbeamertemplate{navigation symbols}{}
  \setbeamertemplate{caption}[numbered]
} 

\usepackage[french]{babel}
\usepackage[utf8x]{inputenc}
\usepackage{fancyvrb}
\usepackage{graphicx}
\usepackage{url}

\author{David Wong
  \and Jacques Monin
  \and Hugo Bonnin}

\title{Implémentation et Analyse d'une White-box du DES}

\institute{Université de Bordeaux}

\date{2014}

\begin{document}

\begin{frame}
  \titlepage
\end{frame}


% \begin{frame}{Plan}
%   \tableofcontents
% \end{frame}

\section{A quoi ça sert ?}

\begin{frame}{A quoi ça sert ?}

  \begin{figure}[h]
    \centering
    \includegraphics[scale=0.6]{./images/kezako.jpg}
  \end{figure}

\end{frame}

\subsection{Man In The Middle}

\begin{frame}{Base de la cryptographie}

  \begin{figure}[h]
    \centering
    \includegraphics[scale=0.50]{./images/alice_bob.png}
  \end{figure}

\end{frame}

\subsection{Man At The End}

\begin{frame}[fragile]{Man At The End}

  \begin{Verbatim}[samepage=true]
    .-----------------. 
    |    ATTAQUANT    |
    |  .-----------.  |  
    |  |           |  |  
    |  | PROGRAMME |  | 
    |  |           |  | 
    |  '-----------'  |
    |                 |
    '-----------------'
  \end{Verbatim}
\end{frame}

\subsection{Exemples}

\begin{frame}{Exemples}

  \begin{figure}[h]
    \centering
    \includegraphics[scale=0.50]{./images/drms.png}
  \end{figure}

\end{frame}

\section{Whitebox}

\subsection{Définition}

\begin{frame}{Définition}

  \begin{figure}[h]
    \centering
    \includegraphics[scale=0.30]{./images/blackbox.png}
  \end{figure}

  \begin{figure}[h]
    \centering
    \includegraphics[scale=0.30]{./images/whitebox.png}
  \end{figure}

\end{frame}

\subsection{DES}
\begin{frame}{Algorithme DES}
  \begin{itemize}
  \item Le but est de transformer toutes ces opérations
    \begin{figure}[h]
      \centering
      \includegraphics[scale=0.45]{./images/des_slides.png}
      \label{fig:DES-round}
    \end{figure}
  \end{itemize}

\end{frame}

\subsection{Github}
\begin{frame}{Github}
  \begin{itemize}
  \item DES : \url{www.github.com/mimoo/DES}
  \item WHITEBOX-DES : \url{www.github.com/mimoo/whiteboxDES}
  \end{itemize}

  \begin{figure}[h]
    \centering
    \includegraphics[scale=0.30]{./images/github.png}
  \end{figure}
\end{frame}

\section{Concepts}

\subsection{Partial Evaluation}
\begin{frame}{Partial evaluation}
  \begin{itemize}
  \item Regrouper le XOR entre le bloc et la clé avec l'opération de substitution.
  \item On peut ensuite pré-calculer toutes les sorties possibles de cette opération.
  \item Les tables créées sont les seules du programme à être modifiées lorsqu'une nouvelle clé est utilisée.
  \end{itemize}
\end{frame}


\subsection{Tabularization}

\begin{frame}[fragile]{Tabularization}
  \begin{figure}[h]
    \centering
    \includegraphics[scale=0.60]{images/tabu.png}
    \caption{Tabularisation}
    \label{fig:keygen}
  \end{figure}
\end{frame}

\begin{frame}{Transformation}
  \begin{figure}[h]
    \centering
    \includegraphics[scale=0.20]{images/etape_1_avant.png}

  \end{figure}

  \begin{figure}[h]
    \centering
    \includegraphics[scale=0.20]{images/etape_1_apres.png}
  \end{figure}
\end{frame}


\begin{frame}{Décomposition de Matrice}
  \begin{figure}[h]
    \centering
    \includegraphics[scale=0.50]{images/decompo_matrice.png}
    \caption{Décomposition de Matrice}
    \label{fig:keygen}
  \end{figure}
\end{frame}

\subsection{Input/Output Encoding}

\begin{frame}{Input/Output Encoding}
  \begin{figure}[h]
    \centering
    \includegraphics[scale=0.5]{images/encoding.png}
    \caption{Encoding}
    \label{fig:keygen}
  \end{figure}
\end{frame}

\section{Concepts secondaires}

\begin{frame}[fragile]{Concepts secondaires}
  \begin{Verbatim}[samepage=true]
    *********************************************
    *              state 2 (96 bits)            *
    *********************************************
    |      |      |                       |
    v      v      v          ...          v
    
    ?????????????????????????????????????????????
    
    |      |      |          ...          |
    v      v      v                       v
    *********************************************
    *              state 3 (96 bits)            *
    *********************************************
  \end{Verbatim}
\end{frame}

\subsection{Randomization}

\begin{frame}{Randomization}
  \begin{figure}[h]
    \centering
    \includegraphics[scale=0.4]{./images/randomization.png}
  \end{figure}
\end{frame}

\subsection{Mixing Bijection}

\begin{frame}{Mixing Bijection}

  \begin{center}

    {\tiny
      000000000000000000000000000000010000000000000000000000001000000000000000000000000000000000000000
      100000000000000000000000000000000000000000000001000000000000000000000000000000000000000000000000
      010000000000000000000000000000000000001000000000000000000000000000000000000000000000000000000000
      001000000000000000000000000000000000000000000000000100000000000000000000000000000000000000000000
      000100000000000000000000000000000000000000000000000010000000000000000000000000000000000000000000
      000010000000000000000000000000000000000000000000000000000000100000000000000000000000000000000000
      000100000000000000000000000000000000000000000000000010000000000000000000000000000000000000000000
      000010000000000000000000000000000000000000000000000000000000100000000000000000000000000000000000
      000001000000000000000000000000000000000000010000000000000000000000000000000000000000000000000000
      000000100000000000000000000000000000000000000000000000000001000000000000000000000000000000000000
      000000010000000000000000000000000000000000000000100000000000000000000000000000000000000000000000
    }

    $G^{-1} \cdot (G \cdot M_1)$ où $G \cdot M_1$
  \end{center}

\end{frame}

\subsection{Bypass}

\begin{frame}[fragile]{Bypass}
  \begin{figure}[h]
    \centering
    \includegraphics[scale=0.4]{./images/state2.png}
  \end{figure}

  \begin{itemize}
  \item On empêche l'identification facile des opérations
  \item On rajoute des bits en entrée et en sortie
  \end{itemize}

\end{frame}


\subsection{Combined Function Encoding}

\begin{frame}{Combined Function}
  \begin{figure}[h]
    \centering
    \includegraphics[scale=0.4]{./images/etape2.png}
  \end{figure}
  \begin{center}
    $(P||Q)(input_P||input_Q)$.
  \end{center}
\end{frame}

\subsection{Split-Path Encoding}

\begin{frame}[fragile]{Split-Path Encoding}
  \begin{Verbatim}[samepage=true]
    Entrée                 S-box              Sortie
                   .--------------------.
    0011||0010 --> |...| 0011||0010 |...|
                   |----------------|---|
                   |...|    0001    |...| --> 0001
                   '--------------------'
                             | 
                             v
                   .--------------------.
    0011||0010 --> |...| 0011||0010 |...|
                   |----------------|---|
                   |...| 0001||xxxx |...| --> 0001||1001
                   '--------------------'
  \end{Verbatim}
  
\end{frame}

\subsection{External Encoding}

\begin{frame}{External Encoding}

  \begin{center}
    \begin{itemize}
    \item Appliquer deux bijections à l'entrée et la sortie de DES
    \item $Whitebox = E \circ DES(input) \circ G$
    \end{itemize}
  \end{center}

\end{frame}

\section{Conclusion}

\begin{frame}{Conclusion}
  
  \begin{itemize}
  \item Beaucoup d'effort pour d'autres solutions (API, clés publiques)
  \item Taille importante
  \item La non-connaissance des algorithmes est ``trop'' importante.
  \item Utilisé profesionnellement
  \end{itemize}

  \begin{figure}[h]
    \centering
    \includegraphics[scale=0.6]{./images/conclusion.png}
  \end{figure}

\end{frame}

\end{document}
